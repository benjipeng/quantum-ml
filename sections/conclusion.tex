This review has examined the growing impact of artificial intelligence techniques across the quantum computing stack, from hardware design through algorithm execution. AI methods are demonstrating significant advantages in addressing many of quantum computing's most challenging problems.

Key themes that emerge across applications include:

\begin{itemize}
    \item AI's ability to navigate high-dimensional, non-convex optimization landscapes that are common in quantum computing
    \item The effectiveness of reinforcement learning for discovering counterintuitive control strategies and designs
    \item The promise of generative models for creating efficient quantum circuits and protocols
    \item The importance of incorporating physics-informed constraints and structure into AI models for quantum applications
\end{itemize}

As both AI and quantum computing continue to advance, their synergy will likely accelerate progress toward practical quantum advantage. The incorporation of increasingly sophisticated AI methods into quantum computing research may help overcome current obstacles in scaling quantum systems to fault tolerance. Conversely, the unique challenges posed by quantum computing will continue to drive innovation in AI techniques.

This reciprocal relationship between AI and quantum computing represents an exciting frontier in computer science research, with potential impacts across science, engineering, and society. 