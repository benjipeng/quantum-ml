Error management is essential for reliable quantum computation. This section examines AI approaches for both error correction and mitigation.

\subsection{Neural Decoders for Quantum Error Correction}
Quantum error correction (QEC) detects and corrects errors through syndrome measurements. For a stabilizer code with stabilizer group $\mathcal{S}$, the decoding problem involves:

\begin{enumerate}
    \item Measuring syndrome bits $s_i = \langle \psi | S_i | \psi \rangle$ for stabilizers $S_i \in \mathcal{S}$
    \item Inferring the most likely error $E$ given syndrome pattern $s$: $P(E|s)$
    \item Applying the appropriate recovery operation $R$
\end{enumerate}

Neural network decoders frame this as a classification problem, mapping syndromes to recovery operations:

\begin{equation}
f_\theta: \{0,1\}^m \rightarrow \mathcal{P}^n
\end{equation}

where $m$ is the number of syndrome bits, $n$ is the number of physical qubits, and $\mathcal{P}^n$ is the Pauli group on $n$ qubits.

AI approaches offer several advantages:

\begin{itemize}
    \item \textbf{Speed}: Neural decoders can achieve microsecond-scale decoding latency, critical for real-time error correction
    
    \item \textbf{Adaptability}: Models can be trained on device-specific noise profiles rather than assuming idealized noise models
    
    \item \textbf{Scalability}: Architecture designs like convolutional decoders scale efficiently to larger code distances
\end{itemize}

Recent implementations have demonstrated neural decoders with performance approaching or exceeding traditional algorithms like minimum-weight perfect matching, but with significantly reduced computational overhead.

\subsection{Error Correction Code Discovery}
The search for optimal quantum error correction codes presents a complex discrete optimization problem. AI methods explore this space through:

\begin{itemize}
    \item \textbf{Reinforcement learning}: Agents that iteratively construct stabilizer sets, receiving rewards based on code distance and rate
    
    \item \textbf{Genetic algorithms}: Evolutionary approaches that "breed" promising codes to discover improved variants
    
    \item \textbf{Autoencoder architectures}: Neural networks that discover efficient encodings into protected subspaces
\end{itemize}

These approaches have discovered codes with improved distance properties and efficient encoding/decoding circuits tailored to specific hardware constraints.

\subsection{Error Mitigation Techniques}
For near-term quantum devices without full error correction, error mitigation techniques improve computational results. AI enhances these approaches through:

\begin{itemize}
    \item \textbf{Zero-noise extrapolation}: Machine learning models can predict zero-noise limits from measurements at different noise levels:
    
    \begin{equation}
    \langle O \rangle_{\text{ideal}} \approx f_\theta(\langle O \rangle_{\lambda_1}, \langle O \rangle_{\lambda_2}, \ldots, \langle O \rangle_{\lambda_k})
    \end{equation}
    
    where $\langle O \rangle_{\lambda_i}$ represents expectation values at noise scale $\lambda_i$.
    
    \item \textbf{Quantum subspace expansion}: Neural networks can identify optimal subspaces that minimize error effects
    
    \item \textbf{Measurement error mitigation}: ML models learn calibration matrices that correct for readout errors:
    
    \begin{equation}
    \vec{p}_{\text{true}} = A^{-1} \vec{p}_{\text{measured}}
    \end{equation}
    
    where $A$ is the calibration matrix and $\vec{p}$ represents measurement outcome probabilities
\end{itemize}

\subsection{Adaptive Measurement and Tomography}
AI techniques enable more efficient quantum state characterization through:

\begin{itemize}
    \item \textbf{Bayesian experimental design}: Selecting optimal measurements to maximize information gain about quantum states
    
    \item \textbf{Compressed sensing}: Reconstructing quantum states from an incomplete set of measurements, formulated as:
    
    \begin{equation}
    \min_\rho \|\rho\|_1 \quad \text{subject to} \quad \|y - \mathcal{A}(\rho)\|_2 < \epsilon
    \end{equation}
    
    where $y$ represents measurement outcomes, $\mathcal{A}$ is the measurement operator, and $\rho$ is the density matrix
    
    \item \textbf{Neural tomography}: Networks that directly map measurement statistics to density matrix estimates
\end{itemize}

The integration of these error correction and mitigation techniques, enhanced by AI, enables more reliable computation on both current and future quantum hardware. 