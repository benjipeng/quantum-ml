Quantum error correction (QEC) is essential for large-scale fault-tolerant quantum computing. This section explores how AI methods are advancing both the decoding of error syndromes and the discovery of new quantum error correction codes.

\subsection{Decoding}
Fast and accurate decoding of error syndromes is crucial for practical quantum error correction. Neural network decoders have shown promise in:

\begin{itemize}
    \item Reducing decoding latency compared to traditional algorithms
    \item Handling complex noise models beyond the standard depolarizing assumption
    \item Maintaining high performance across different code sizes
    \item Integrating with hardware-level constraints
\end{itemize}

\subsection{Code Discovery}
The search for optimal quantum error correction codes with desirable properties presents a complex optimization challenge. AI methods, particularly reinforcement learning and genetic algorithms, have begun to assist in:

\begin{itemize}
    \item Discovering codes with improved distance properties
    \item Finding codes tailored to specific hardware noise profiles
    \item Identifying codes with efficient encoding/decoding circuits
    \item Exploring the space of non-stabilizer codes
\end{itemize} 