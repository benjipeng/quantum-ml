Controlling quantum systems with high fidelity presents unique challenges that AI methods are particularly well-suited to address. This section examines how AI is improving quantum device control and optimization.

Reinforcement learning has emerged as a powerful approach for quantum control optimization, allowing the discovery of control sequences that outperform those designed using conventional methods. Key applications include:

\begin{itemize}
    \item Optimizing single and multi-qubit gate implementations
    \item Developing robust control sequences that maintain performance despite system variations
    \item Real-time adaptive control based on measurement feedback
    \item Learning control strategies for complex many-body quantum systems
\end{itemize}

Recent work has demonstrated how reinforcement learning agents can break constraints imposed by adiabatic quantum control \cite{ding2021breaking}, potentially opening new routes to high-fidelity quantum operations. 

Quantum systems require precise control to maintain coherence and execute operations with high fidelity. AI techniques offer significant advantages for both offline control optimization and real-time error management.

\subsection{Optimal Control Theory}
Quantum optimal control seeks time-dependent control fields $\{c_j(t)\}$ that drive a quantum system governed by the Hamiltonian:

\begin{equation}
H(t) = H_0 + \sum_j c_j(t) H_j
\end{equation}

where $H_0$ is the drift Hamiltonian and $\{H_j\}$ are control Hamiltonians. The goal is to evolve the system from an initial state $\rho_0$ to a target state $\rho_T$ or to implement a target unitary $U_T$.

Traditional approaches like GRAPE (Gradient Ascent Pulse Engineering) use gradient information to optimize control parameters. AI methods extend these capabilities through:

\begin{itemize}
    \item \textbf{Model-free optimization}: Reinforcement learning agents can discover high-fidelity control sequences without explicit system models \cite{bukov2018reinforcement}
    
    \item \textbf{Robustness to uncertainty}: Neural network controllers can maintain performance despite variations in system parameters
    
    \item \textbf{Breaking adiabatic constraints}: RL approaches have discovered control protocols that exceed the speed limits of adiabatic control \cite{ding2021breaking}
\end{itemize}

\subsection{Real-time Adaptive Control}
Real-time control adjusts operations based on continuous measurement feedback. For a partially observable quantum system, the control problem becomes a partially observable Markov decision process (POMDP) where:

\begin{itemize}
    \item States represent quantum density matrices $\rho$
    \item Actions are control operations
    \item Observations are measurement outcomes with probability distributions determined by quantum mechanics
\end{itemize}

Deep reinforcement learning strategies for this POMDP can be formulated using recurrent neural networks that maintain internal representations of quantum state estimates. The resulting controllers can:

\begin{equation}
\pi(a_t | o_1, o_2, \ldots, o_t) = \text{RNN}(o_t, h_{t-1})
\end{equation}

where $o_t$ are observations, $a_t$ are actions, and $h_t$ is the hidden state of the RNN.

\subsection{Multi-qubit Operation Scheduling}
Executing multiple operations across a quantum processor requires sophisticated scheduling to minimize crosstalk and maximize parallelism. Machine learning approaches formulate this as a constrained optimization problem:

\begin{equation}
\min_{\{t_i\}} \sum_i w_i t_i \quad \text{subject to constraints on overlapping operations}
\end{equation}

where $t_i$ represents the start time of operation $i$ and $w_i$ its priority weight.

AI schedulers can consider hardware-specific constraints such as:
\begin{itemize}
    \item Control line bandwidth limitations
    \item Crosstalk between neighboring qubits
    \item Time-dependent noise characteristics
    \item Variability in operation fidelities
\end{itemize}

\subsection{In-situ Calibration and Drift Compensation}
Quantum systems exhibit parameter drift over time, requiring continuous recalibration. AI techniques enable efficient in-situ calibration through:

\begin{itemize}
    \item \textbf{Bayesian experimental design}: Selecting optimal calibration experiments to maximize information gain
    
    \item \textbf{Online learning}: Continuously updating system models based on observed performance
    
    \item \textbf{Predictive maintenance}: Forecasting parameter drift and scheduling preventive recalibration
\end{itemize}

These control techniques significantly enhance quantum computer performance by optimizing hardware operation at multiple timescales - from nanosecond pulse shaping to millisecond measurement feedback and hour-to-hour drift compensation. 