Quantum computing promises computational advantages for specific problems by harnessing quantum mechanical phenomena such as superposition and entanglement \cite{alexeev2021quantum}. As quantum hardware advances from noisy intermediate-scale quantum (NISQ) devices toward fault-tolerant systems, numerous technical challenges emerge across the quantum computing stack - from device physics to algorithm implementation.

Artificial intelligence offers powerful tools for addressing these challenges. The complex, high-dimensional, and non-linear nature of quantum systems makes them particularly amenable to AI approaches \cite{dunjko2023artificial}. Machine learning techniques excel at recognizing patterns in multidimensional spaces and optimizing complex functions - capabilities directly applicable to quantum computing problems.

This article explores applications of AI techniques for enhancing quantum computing development and operation. We focus on how AI accelerates quantum computing research and implementation rather than on quantum computing's potential future impact on AI itself. The content addresses the complete quantum computing workflow, from fundamental AI methods relevant to quantum applications through hardware optimization, circuit synthesis, control systems, error management, and error correction strategies. For each area, we present the mathematical foundations where appropriate and examine how AI techniques provide novel solutions to quantum computing challenges.

\begin{figure}[!t]
\centering
\begin{tikzpicture}[node distance=1.2cm, auto, >=latex', scale=0.7, transform shape]
    % Define nodes
    \node [rectangle, rounded corners, minimum width=12cm, minimum height=1.5cm, draw=black, fill=gray!10] (qh) {Quantum Hardware};
    \node [rectangle, rounded corners, minimum width=12cm, minimum height=1.5cm, draw=black, fill=gray!10, above of=qh, node distance=2cm] (ctl) {Quantum Control \& Error Management};
    \node [rectangle, rounded corners, minimum width=12cm, minimum height=1.5cm, draw=black, fill=gray!10, above of=ctl, node distance=2cm] (circ) {Circuit Synthesis \& Optimization};
    \node [rectangle, rounded corners, minimum width=12cm, minimum height=1.5cm, draw=black, fill=gray!10, above of=circ, node distance=2cm] (error) {Error Correction \& Mitigation};
    \node [rectangle, rounded corners, minimum width=12cm, minimum height=1.5cm, draw=black, fill=gray!10, above of=error, node distance=2cm] (app) {Quantum Applications};
    
    % Define AI box
    \node [rectangle, rounded corners, minimum width=3cm, minimum height=11cm, draw=aigreen!80, thick, fill=aigreen!10, left of=ctl, node distance=7.5cm] (ai) {AI Methods};
    
    % Define arrows
    \draw[->, thick, aigreen] (ai.east) -- ++(1,0) |- (qh.west);
    \draw[->, thick, aigreen] (ai.east) -- ++(1,0) |- (ctl.west);
    \draw[->, thick, aigreen] (ai.east) -- ++(1,0) |- (circ.west);
    \draw[->, thick, aigreen] (ai.east) -- ++(1,0) |- (error.west);
    \draw[->, thick, aigreen] (ai.east) -- ++(1,0) |- (app.west);
    
    % Labels for quantum computing layers
    \node[right] at (qh.west) {\textbf{Hardware Design \& Optimization}};
    \node[right] at (ctl.west) {\textbf{Pulse Design \& Real-time Control}};
    \node[right] at (circ.west) {\textbf{Algorithm Implementation}};
    \node[right] at (error.west) {\textbf{Noise Mitigation \& Correction}};
    \node[right] at (app.west) {\textbf{Quantum Algorithm Design}};
    
    % AI methods labels
    \node[align=left] at (ai) {
        \textbf{Neural Networks}\\[0.2cm]
        \textbf{Reinforcement Learning}\\[0.2cm]
        \textbf{Generative Models}\\[0.2cm]
        \textbf{Transfer Learning}
    };
    
\end{tikzpicture}
\caption{Integration of AI methods across the quantum computing stack. AI techniques enhance each layer of quantum computing development, from hardware design to application implementation. The bidirectional relationship enables advances in both fields, with quantum computing challenges driving AI innovation.}
\label{fig:ai_quantum_stack}
\end{figure}

\subsection{AI Paradigms for Quantum Computing}

Several key AI paradigms have demonstrated particular relevance to quantum computing challenges. Supervised learning models, trained on labeled data, can predict properties of quantum systems with remarkable accuracy. These methods have proven effective for tasks such as predicting the ground state energies of molecular systems or estimating the fidelity of quantum operations under noise.

Unsupervised learning techniques discover structure in quantum data without explicit labels. Such approaches have been successfully applied to identify patterns in experimental measurement results and to cluster quantum states according to their entanglement properties \cite{janiesch2021machine}.

Reinforcement learning (RL) has emerged as a particularly powerful paradigm for quantum computing \cite{arulkumaran2017deep, shakya2023reinforcement}. By formulating quantum control and optimization as sequential decision processes, RL agents learn optimal policies through direct interaction with quantum systems or their simulations. This approach has proven especially valuable for discovering non-intuitive control strategies that outperform conventional methods.

Deep learning architectures, with their ability to extract hierarchical features from complex data \cite{lecun2015deep}, provide the foundation for many quantum applications. These neural networks can process the high-dimensional data associated with quantum states and processes, enabling more efficient representation and manipulation of quantum information.

Generative models represent another crucial AI approach for quantum computing. These architectures create novel quantum circuits, control sequences, and error correction codes by learning probability distributions over complex quantum objects \cite{bernardo2007generative}. Recent advances in generative models have demonstrated their ability to discover quantum protocols that match or exceed those designed by human experts.

\subsection{Structure of this Article}

The remainder of this article is organized according to the quantum computing workflow. Section 2 examines fundamental AI techniques and their mathematical formulations relevant to quantum computing applications. Section 3 explores quantum hardware optimization, including system identification, device design, and pulse optimization. Section 4 addresses circuit synthesis and optimization, focusing on AI methods for transforming abstract algorithms into efficient implementations. Section 5 covers quantum control and error management during execution, while Section 6 explores AI approaches for error correction and mitigation. Finally, Sections 7 and 8 discuss emerging research directions and summarize the key insights regarding the synergy between AI and quantum computing. 