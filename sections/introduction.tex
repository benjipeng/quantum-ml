Quantum computing promises computational advantages for specific problems by harnessing quantum mechanical phenomena such as superposition and entanglement \cite{alexeev2021quantum}. As quantum hardware advances from noisy intermediate-scale quantum (NISQ) devices toward fault-tolerant systems, numerous technical challenges emerge across the quantum computing stack - from device physics to algorithm implementation.

Artificial intelligence offers powerful tools for addressing these challenges. The complex, high-dimensional, and non-linear nature of quantum systems makes them particularly amenable to AI approaches \cite{dunjko2023artificial}. Machine learning techniques excel at recognizing patterns in multidimensional spaces and optimizing complex functions - capabilities directly applicable to quantum computing problems.

This article explores applications of AI techniques for enhancing quantum computing development and operation. We focus on how AI accelerates quantum computing research and implementation rather than on quantum computing's potential future impact on AI itself. The content addresses the complete quantum computing workflow:

\begin{itemize}
    \item Fundamental AI methods relevant to quantum computing applications
    \item Hardware optimization, including device characterization and control pulse design
    \item Circuit synthesis and optimization for efficient quantum algorithm implementation
    \item Control systems and error management during quantum execution
    \item Error correction and mitigation strategies for improving computational results
\end{itemize}

For each area, we present the mathematical foundations where appropriate and examine how AI techniques provide novel solutions to quantum computing challenges.

AI approaches for quantum computing span several key paradigms:

\begin{itemize}
    \item \textbf{Supervised learning}: Models trained on labeled data to predict properties of quantum systems.
    \item \textbf{Unsupervised learning}: Techniques that discover structure in quantum data without explicit labels.
    \item \textbf{Reinforcement learning (RL)}: Agents that learn optimal control policies through interaction with quantum systems \cite{arulkumaran2017deep, shakya2023reinforcement}.
    \item \textbf{Deep learning}: Neural networks extracting hierarchical features from quantum data \cite{lecun2015deep}.
    \item \textbf{Generative models}: Architectures creating novel quantum circuits, control sequences, or error correction codes \cite{bernardo2007generative}.
\end{itemize}

These approaches, often in combination, are transforming quantum computing research by accelerating development cycles, discovering non-intuitive solutions, and extracting insights from complex quantum data. 